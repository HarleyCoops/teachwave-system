\section*{Key Concept: Measuring Returns over Time (Time-Weighted vs. Money-Weighted)}

In performance evaluation, you often need to distinguish between the flows of capital into and out of a portfolio (external cash flows) and the actual returns generated by the underlying investments. Two primary return metrics are:

\[
\textbf{Time-Weighted Return (TWR)}: \quad TWR = \left(\prod_{i=1}^{n} (1 + R_i)\right)^{\frac{1}{n}} - 1
\]

\[
\textbf{Money-Weighted Return (MWR)}: \quad \text{MWR} \text{ is the internal rate of return (IRR) that sets the net present value of cash flows to zero.}
\]

The TWR is most appropriate when you want to measure a manager's performance irrespective of the timing and size of cash flows. The MWR incorporates the impact of cash flow timing and magnitude from the investor's perspective. To compare both returns on a risk-adjusted basis, common measures include the Sharpe ratio:

\[
\textbf{Sharpe Ratio} = \frac{R_p - R_f}{\sigma_p}
\]

where \(R_p\) is the portfolio return, \(R_f\) is the risk-free rate, and \(\sigma_p\) is the standard deviation of the portfolio returns.

\section*{Practice Question}
10. A sovereign wealth fund is in the process of re-evaluating its historical investment performance over a three-year period marked by geopolitical events impacting cash flows. The candidate must compute a series of returns including compounded annual growth rates, money-weighted returns, and their associated risk-adjusted measures, and critically discuss which approach best isolates market performance from cash flow effects.

Assume the following data for a three-year horizon (all figures in millions):
\begin{itemize}
\item Initial investment at the start of Year 1: \$100
\item End of Year 1 portfolio value: \$108, then an additional cash inflow of \$20 (immediately reinvested)
\item End of Year 2 portfolio value: \$130, followed by a \$10 cash outflow
\item End of Year 3 final portfolio value: \$140
\end{itemize}
Over this three-year span, the annualized risk-free rate \(R_f\) is 2\%, and the standard deviation of the portfolio's annual returns is 12\%.

\begin{enumerate}
\item Using the above information, calculate the three-year \textbf{time-weighted return} (compounded annual growth rate).
\item Calculate the \textbf{money-weighted return} (internal rate of return) over the same period.
\item Compute the \textbf{Sharpe ratio} using your time-weighted return and the given risk-free rate and standard deviation.
\item Finally, recompute the \textbf{Sharpe ratio} using your money-weighted return and interpret the difference numerically (no verbal commentary requested, only provide the calculated value).
\end{enumerate}

\section*{Solution and Explanation}

\subsection*{Part 1: Time-Weighted Return (TWR)}
\[
\text{Step 1: Calculate sub-period returns.}
\]
\[
\text{Year 1 Return} = \frac{\$108 - \$100}{\$100} = 0.08 \quad (8\%)
\]
\[
\text{New investment of \$20 at end of Year 1 does not affect Year 1 return.}
\]
\[
\text{Start of Year 2 investment base} = \$108 + \$20 = \$128
\]
\[
\text{Year 2 Return} = \frac{\$130 - \$128}{\$128} = 0.015625 \quad (1.5625\%)
\]
\[
\text{Outflow of \$10 at end of Year 2 does not affect Year 2 return.}
\]
\[
\text{Start of Year 3 investment base} = \$130 - \$10 = \$120
\]
\[
\text{Year 3 Return} = \frac{\$140 - \$120}{\$120} = 0.1667 \quad (16.67\%)
\]

\[
\text{Step 2: Calculate TWR as the geometric mean of sub-period returns.}
\]
\[
(1 + R_1) = 1.08,\quad (1 + R_2) = 1.015625,\quad (1 + R_3) = 1.1667
\]
\[
\text{TWR} = \Big(1.08 \times 1.015625 \times 1.1667\Big)^{\frac{1}{3}} - 1
\]
Numerically:
\[
= (1.27459)^{\frac{1}{3}} - 1 \approx 0.081 \quad (8.1\%)
\]

\subsection*{Part 2: Money-Weighted Return (MWR / IRR)}
\[
\text{Define cash flows and solve for IRR:}
\]
\[
\text{At time }t=0: -\$100
\]
\[
\text{At time }t=1: +(\$108 - \$100) = +\$8 \quad \text{and inflow of \$20, net is } -\$20 \text{ not a separate return component, so overall is }
\]
\[
\text{Time }t=1 \text{ net flow }= +\$108 - \$100 + (-\$20)\text{? This must be structured carefully.}
\]
Alternatively, treat the total else as:
\[
\text{Time }0: -\$100
\]
\[
\text{Time }1: -\$20 \quad \text{(since the \$108 is the new total, the net external flow is \$20 in)}
\]
\[
\text{Time }2: +(-\$10) \quad \text{(because \$10 is withdrawn)}
\]
\[
\text{Time }3: +\$140 \quad \text{(final liquidation)}
\]
Solve for \(r\) in:
\[
-\$100 + \frac{-\$20}{(1+r)} + \frac{-\$10}{(1+r)^2} + \frac{\$140}{(1+r)^3} = 0
\]
Using a financial calculator or iterative method yields approximately:
\[
r \approx 6.22\%
\]
(annualized MWR)

\subsection*{Part 3: Sharpe Ratio Using TWR}
\[
R_p = 8.1\%, \quad R_f = 2\%, \quad \sigma_p = 12\%
\]
\[
\text{Sharpe Ratio}_{\text{TWR}} = \frac{0.081 - 0.02}{0.12} = 0.5083
\]

\subsection*{Part 4: Sharpe Ratio Using MWR}
\[
R_p = 6.22\%, \quad R_f = 2\%, \quad \sigma_p = 12\%
\]
\[
\text{Sharpe Ratio}_{\text{MWR}} = \frac{0.0622 - 0.02}{0.12} = 0.3517
\]

\begin{infobox}[Christian's Thoughts]

It helps to remember that the time-weighted return (TWR) is best for assessing the manager's skill because it neutralizes external cash flows, while the money-weighted return (MWR) reflects your actual experience as an investor in the presence of contributions or withdrawals. When comparing these returns on a risk-adjusted basis, knowing which figure to use depends on whether you're measuring investment decisions independent of cash flow timing (TWR) or your actual internal rate of growth (MWR). If you love deeper insights, keep practicing both calculations until they feel natural!

-christian

\end{infobox}