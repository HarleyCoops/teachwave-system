\section*{\(\text{Key Concept: Calculate and Interpret Different Approaches to Return Measurement Over Time and Describe Their Appropriate Uses}\)}

\(\text{In evaluating investment performance over multiple subperiods, two common methods for computing returns are the arithmetic mean return and the geometric mean return.}\)

\(\textbf{Arithmetic Mean Return (Nominal Annualized):}\)
\[
\text{Arithmetic Mean, } \bar{r} \;=\;\frac{r_{1}+r_{2}+\cdots+r_{n}}{n}
\]
\(\text{When subperiods are of equal length (e.g., months or quarters), the average subperiod return can be multiplied appropriately to annualize it.}\)

\(\textbf{Geometric Mean Return:}\)
\[
\text{Geometric Mean, } r_{\text{geo}} \;=\;\left(\prod_{i=1}^{n}(1+r_{i})\right)^{\frac{1}{n}}-1
\]
\(\text{The geometric mean captures the effect of compounding and is typically more appropriate for multi-period investment evaluations.}\)

\section*{\(\text{Practice Question}\)}
\(\text{2. An asset management firm is reviewing the annual performance of a mutual fund that experienced several mid-year dividend reinvestments and periodic external cash flows. Candidates are asked to compute the nominal annualized return using simple arithmetic returns, adjust these for compounding effects to determine geometric returns, and interpret the differences between these approaches.}\)

\(\text{Assume you observed the following 6-month returns for the mutual fund: 6\% over the first half of the year and 10\% over the second half of the year.}\)

\begin{enumerate}
\item \(\text{Calculate the nominal annualized return using the arithmetic approach based on these two subperiod returns.}\)
\item \(\text{Calculate the geometric annual return using the same subperiod returns.}\)
\item \(\text{Compute the numerical difference between the arithmetic annualized return and the geometric annual return.}\)
\end{enumerate}

\section*{\(\text{Solution and Explanation}\)}

\subsection*{\(\text{Part 1: Arithmetic Annualized Return}\)}
\(\text{Let } r_{1}=0.06 \text{ and } r_{2}=0.10 \text{ for the two 6-month periods. The arithmetic mean of subperiod returns is:}\)

\[
\bar{r} 
=
\frac{r_{1}+r_{2}}{2}
=
\frac{0.06 + 0.10}{2}
=
0.08
=
8\%
\]

\(\text{Because each holding period is half a year, the nominal annualized return under the arithmetic approach is } 8\%\times 2=16\%. \)

\subsection*{\(\text{Part 2: Geometric Annual Return}\)}
\(\text{Using compounding, the total one-year growth factor is:}\)

\[
(1 + r_{1})(1 + r_{2}) = (1.06)(1.10) = 1.166
\]

\(\text{Hence, the geometric annual return is:}\)

\[
1.166 - 1 = 0.166 = 16.6\%
\]

\subsection*{\(\text{Part 3: Difference Between Arithmetic and Geometric}\)}
\(\text{The arithmetic annualized return is }16\%\text{, while the geometric annual return is }16.6\%\text{. The numerical difference is:}\)

\[
16.6\% - 16.0\% = 0.6\%
\]

\begin{infobox}[Christian's Thoughts]

\(\text{I always remind myself that arithmetic returns are simple averages of the subperiod rates, while geometric returns show the actual compound growth. The geometric method is particularly crucial when returns vary across periods or when negative returns appear, since compounding truly captures the proportional impact of gains and losses on portfolio value. Test yourself often by comparing your arithmetic vs. geometric calculations to see how compounding can move the final annual figure.}\)

\(-\text{christian}

\end{infobox}